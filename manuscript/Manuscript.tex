\documentclass[12pt]{article}
\usepackage{graphicx}
\usepackage[margin=1in]{geometry}
\usepackage{setspace}

\title{Predicting the Housing Market using \\ Linear Regression}

\author{Garrick Ho\\
  Jun Yan\\[2ex]
  Department of Statistics\\
  University of Connecticut\\
}

\begin{document}

\maketitle
\doublespace

\begin{abstract}
    With the housing market fluctuating every year, many real estate owners and businesses face a problem with gaining an accurate representation of the future of the housing market. Lots of people do not know much about the market, so they will go to real estate owners or businesses to learn more about the market. And then see what the market is at and if it is the right time to buy a house. If real estate owners and businesses are able to gain more accuracy in predicting the housing market, they will be able to help more people and possibly gain more business. Not only can this benefit the consumers, the builders are able to move with the market and have a better understanding of it. This paper aims to construct a robust linear regression model by using machine learning techniques to predict housing costs accurately. By leveraging natural language processing, the linear regression model analyzes home property descriptions and extracts vital insights that significantly contribute to precise pricing predictions. This paper ends with an explanation of the limitations of this study and potential future works that can be done.

\bigskip
\noindent{\sc Keywords}:
housing market;
natural language processing;
linear regression;
model predictions


\end{abstract}

\section{Introduction}

\section{Data}

\section{Methods}

\begin{equation}
  \label{eq:Multi}
  Y = \beta_{0} + \beta_{1}X_{1} + \beta_{2}X_{2} + ... + \beta_{n}X_{n} + \epsilon
\end{equation}

Equation~\ref{eq:Multi} is the Multiple Linear Regression model being used. \(Y\) is the dependent variable, also known as the value that is being predicted. In this case, it would be the cost of a house. \(X_{n}\) is the independent variable, which is the characteristics of the house that are used to predict the house cost. \(\beta_{n}\) is the average amount by which the dependent variable increases and decreases depending on when the independent variable increases one standard deviation. When all the other independent variables are held constant. \(\beta_{0}\) represents the value of the dependent variable when all the independent variables are equal to zero. \(\epsilon\) is the error term which represents the margin of error within the linear regression model. 

\section{Simulation}

\section{Application}

\section{Discussion and Conclusion}

\section{Appendix}

\section{References}


\end{document}
